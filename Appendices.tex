\section{Appendices}
\label{sec:app}

\subsection{Cloud Computing SWOT Analysis}
\label{ssec:swot}
\subsubsection{Cloud service provider}

\renewcommand{\arraystretch}{1.5}
\begin{longtable}{ | p{0.5\linewidth} | p{0.5\linewidth} | } 
\caption{SWOT Cloud Service Providers} \\
 \hline
\textbf{Strengths} & \textbf{Weaknesses}\\ 
\hline
\textit{Scalability}: Using a cloud service enables effortless scaling & \textit{Service Outages}: They do not happen often, but when they happen they are out of the customers control \\ 
\textit{Lower Costs}: Paying only for what the customer uses and not having to worry about maintenance drives down costs & \textit{Longer Upload/Download times}:  Compared to an in-house server it will take longer to move large files, as the internet speed is the limiting factor\\
\textit{Lower Capital Expense}: Since the customer is not the one buying the hardware & \\ 
 \textit{Global Connectivity}: Cloud service providers have a global network of servers and thus the customer can offer his clients a fast connection to a local server & \\
 \textit{Security}: Cloud service providers have invested heavily into security, since their reputation would be at stake if a breach happened. The customer will always have cutting edge security from a technical standpoint. & \\
 \textit{Integration}: Since the service is offered as a platform, the customer has access to other services within the providers ecosystem. Big providers like AWS or Microsoft Azure offer an ever expanding selection of services apart from pure server hosting & \\
 \hline
 \textbf{Opportunities} & \textbf{Threats}  \\ 
 \hline
\textit{Technological Advancements}: A cloud service provider will always seek to have the best technology to offset themselves from competition, which benefits the customer & \textit{Termination of Service}: It seems very unlikely, but in theory the service provider could go out of business/terminate the service and thus disrupt the business \\
\hline 
\end{longtable}

\subsubsection{In-House Server}
In-House servers for rendering purposes can be acquired from graphic card manufacturers such as NVIDIA: https://www.nvidia.com/en-us/design-visualization/quadro-servers/rtx/

\begin{longtable}{ | p{0.5\linewidth} | p{0.5\linewidth} | } 
\caption{SWOT In-House Server} \\
 \hline
 \textbf{Strengths} & \textbf{Weaknesses}\\ 
\hline
 \textit{Total Control}: If the customer owns and operates the server, they have complete control over it. They can adjust the server to specifically fit their requirements and thus optimizing performance & \textit{Increased complexity/costs}: Operating a server infrastructure requires experts to administrate and maintain \\ 
\textit{Faster development/response time}: An in-house server is local by nature and thus modifying/fixing things on the server is faster compared to external servers & \textit{Higher Capital Expense}: Since the customer has to buy all the necessary hardware for an in-house server, the upfront investment is higher \\
\textit{Well understood} It is easier for developers to become familiar with and develop in-depth knowledge in a server infrastructure they can easily interact with & \textit{Physical Requirements}: Building an in-house server infrastructure requires not only sufficient physical space, but also auxiliary systems such as cooling, (emergency-) electricity and cabling \\
 \hline
 \textbf{Opportunities} & \textbf{Threats}  \\ 
 \hline
\textit{TBD}: & \textit{Obsolescence}: As the owner, the customer would be responsible to upgrade the system if they want/need new features. This of course comes with more costs \\
\hline 
\end{longtable}

\subsection{Cloud service providers}
The cloud service providers are ranked based on annual revenue from 2018 in Millions of US Dollars \parencite{iaasrevenue}.
\begin{longtable}{ | p{0.1\linewidth} | p{0.2\linewidth} | p{0.3\linewidth} | p{0.4\linewidth} |} 
\caption{Cloud Service Providers by Revenue} \\
\hline
\textbf{Rank} & \textbf{Company} & \textbf{Product} & \textbf{Revenue 2018} \\ 
\hline
1 & Amazon & AWS & 15,495\\
\hline
2 & Microsoft & Azure & 5,038 \\
\hline
3 & Alibaba & Alibaba Cloud & 2,499 \\
\hline
4 & Google & Google Cloud Platform & 1,314\\
\hline
\end{longtable}
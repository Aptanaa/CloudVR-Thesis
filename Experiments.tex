\section{Experiments}

\subsection{Exploratory Research Phase}
After about a month of researching for existing solutions, applying for NVIDIA CloudXR \parencite{cloudxr} and testing the difficulty of different prototype architectures it became clear that making a prototype from scratch is not a feasible task for the duration and team size of the project. This conclusion is based on two major observations:

\paragraph{NVIDIA CloudXR SDK is the only working solution}
During the research a couple of existing products were found, that demonstrated the feasibility of key areas of interest for this project: Low Latency and High Security \parencite{gtc2020esi}. All of the products found were launched shortly before the start or during the graduation, so it is safe to say that the technology and market is currently emerging. Another overlap of these products was that they all used NVIDIA's CloudXR SDK to enable them to stream \acrshort{vr} content. In the conference talks, where these products were presented, CloudXR was always mentioned as the last piece of the puzzle to enable the creation of ambitions cloud \acrshort{vr} projects.

\paragraph{Making a custom Prototype will not provide any benefits}
When experimenting with the \hyperref[fig:pr11]{first} and \hyperref[fig:pr12]{second} prototype design, both in Unreal Engine 4 and Unity 3D, it was possible to quickly create a WebRTC connection from a 'server' machine to a 'client' machine in the same network. However before I ever got to implementing the whole application loop it was already evident that the performance would be a critical factor. As explored in the \nameref{sec:theo} the \acrfull{mtp} is not allowed exceed 20\acrshort{ms}, where as the latency in the prototypes would be atleast around 100\acrshort{ms} normally. This was measured in a local network setup instead of a cloud server and without streaming the significantly bigger \acrshort{vr} resolution of 2160x1200, all of which things that would have impacted the latency negatively even more. Considering this it would be impossible to create and optimize a prototype to undercut the \acrshort{mtp} barrier within the remaining time frame of the project. Additionally it became clear that alternative solutions, like ReactVR \parencite{reactVR} and A-Frame \parencite{aframe}, are not suitable for the use-cases of the stakeholders as they do not allow for the complexity needed as presented in the \nameref{sec:int}. \\

\subsection{Preliminary Conclusion & Next Steps}
At the end of the exploratory research phase I finally got into contact with an NVIDIA employee who was able to grant me access to the CloudXR SDK. I was able to successfully set up the \hyperref[fig:pr0]{pipeline} in the local network of the XR lab within a couple of days. Some initial user testing from myself and volunteering teachers reveals that the latency in the pipeline is not-at-all/barely impacting the user experience. To generate verifiable results the next steps are to increase the pool of test candidates to be able to test for different kinds of user groups (based on age, motion sickness aptitude, etc.). Additionally there will be an attempt to measure \acrshort{mtp} latency. After some discussion with NVIDIA they offered to send us a 'device' which they developed to measure \acrshort{mtp}. If the devices proves to be ineffective or not usable there are other statistics, such as \acrshort{fps}, that can be used to gain a (less precise) measurement of the latency within the pipeline. Lastly it will be explored how to deploy such a system so that it satisfies the security demands of the project.
\section{Literature Review}
\subsection{Modern ($<$5 years old) research and technology}

Within the last decade the cloud computing space has expanded rapidly and with it the possibilities. Today, even individuals can set up an experimental cloud \acrfull{vr} streaming solution from pre-made components \parencite{tayoexe}, such as the service from cloud computing company Shadow \parencite{shadow}. The most recent years have also seen the rise of commercial cloud streaming for gaming services, such as Google Stadia \parencite{stadia}, XBox XCloud \parencite{xcloud} and Nvidia GeForceNow \parencite{geforcenow}, but not without problems, as the services were quickly overwhelmed on launch. However no commercial products for cloud \acrshort{vr} streaming exist at the time of writing, hence the need for this research paper. Modern video compression codecs, like the AV1 codec introduced in 2018 \parencite{av1}, are getting better and better at compressing high-resolution video streams and together with an application like WebRTC \parencite{webRTC} which offers latency optimizations via peer-to-peer networking and more, they lay the foundation for modern cloud streaming applications. With ever increasing performance and optimizations, the feasibility of cloud \acrshort{vr} streaming is just a matter of time.

Research Papers like the ones from \cite{cutcord} or from \cite{mvr} demonstrate the viability and technical feasibility of cloud \acrshort{vr} streaming . They developed innovative solutions to achieve and undercut the 20 \acrfull{ms}  \acrfull{mtp} delay barrier while streaming \acrshort{vr} content. 20 \acrshort{ms} is the agreed upon threshold a frame should have, from receiving the input to displaying the frame on the \acrfull{hmd}, to avoid inducing motion sickness. One such solution is a low latency control loop that streams \acrshort{vr} scenes containing only the user’s \acrfull{fov} and a latency adaptive margin area around the \acrshort{fov}. This allows the clients to render locally at a high refresh rate to accommodate and compensate for the head movements before the next motion update arrives. \parencite{mvr}. Another angle of attack  leverage's the power of parallel rendering, encoding, transmission and decoding, together with a Remote VSync Driven Rendering approach to minimize \acrshort{mtp} latency \parencite{cutcord}. The prototype for this experiment was based on commodity hardware, which further demonstrates the feasibility of cloud \acrshort{vr} streaming.

%\subsection{Older ($>$5 years old) research and technology} 
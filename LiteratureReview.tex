\section{Literature Review}
\subsection{Modern ($<$5 years old) research and technology}

Within the last decade the cloud computing space has expanded rapidly and with it the possibilities. Today, even indivudials can setup an experimental cloud VR streaming setup from premade components \parencite{tayoexe}, such as the service from cloud computing company Shadow \parencite{shadow}. The most recent years have also seen the rise of commercial cloud streaming for gaming services, such as Google Stadia \parencite{stadia}, XBox XCloud \parencite{xcloud} and Nvidia GeForceNow \parencite{geforcenow}, but not without problems as the services were quickly overwhelmed on launch. However no commercial products for cloud VR streaming exist at the time of writing, hence the need for this research paper.

Research Papers like the ones from \cite{cutcord} or from \cite{mvr} show the way to make cloud VR streaming useful enough to consider commercial development. They developed innovative solutions to achive and undercut the 20ms Motion-to-Photon (MTP) barrier while streaming VR content. 20ms is the agreed upon threshould a frame should have, from recieving the input to displaying the frame on the Head Mounted Display (HMD), to avoid inducing motion sickness. One such solution is a low latency control loop that streams VR scenes containing only the user’s Field of View (FoV) and a latency adaptive margin area around the FoV. This allows the clients to render locally at a high refresh rate to accommodate and compensate for the head movements before the next motion update arrives. \parencite{mvr}. Another angle of attack  leverages the power of parallel rendering, encoding, transmission and decoding, together with Remote VSync Driven Rendering to minimize MTP latency \parencite{cutcord}. The prototype for this experiment was based on commodity hardware, which together with the experiments further demonstrates the feasability of cloud VR streaming.

\subsection{Older ($>$5 years old) research and technology} 


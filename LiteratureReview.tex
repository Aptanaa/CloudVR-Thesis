\section{Literature Review}
\label{sec:lit}
\subsection{Modern ($<$5 years old) research and technology}

Within the last decade the cloud computing space has expanded rapidly and with it the possibilities. Today, even individuals can set up an experimental cloud (\acrfull{vr}) gaming streaming solution from pre-made components \parencite{tayoexe} \parencite{clouddesktopguide}. For less experimentally inclined customers, there are complete services, such as the one from cloud computing company Shadow \parencite{shadow} who recently announced a closed beta for their dedicated VR streaming service \parencite{shadowvr}. Other major players in the cloud gaming  scene are Google's Stadia \parencite{stadia}, Microsoft's XBox XCloud \parencite{xcloud} and Nvidia's GeForceNow \parencite{geforcenow}. These services were quickly overwhelmed on launch and faced public scrutiny for failing to living up to the promises. 

Modern video compression codecs, like the AV1 codec introduced in 2018 \parencite{av1}, are getting better and better at compressing high-resolution video streams and together with an application like WebRTC \parencite{webRTC} which offers latency optimizations via peer-to-peer networking and more, they lay the foundation for modern cloud streaming applications. Technologies like Google's Seurat Image\hyp{}Based Scene Simplification System \parencite{seurat} and Shading atlas streaming \parencite{sas} offer even further optimizations in areas other than networking and transmitting data. 

Research Papers like the ones from \cite{cutcord} or from \cite{mvr} demonstrate the viability and technical feasibility of cloud \acrshort{vr} streaming . They developed innovative solutions to achieve and undercut the 20 \acrfull{ms}  \acrfull{mtp} delay barrier while streaming \acrshort{vr} content. 20 \acrshort{ms} is the agreed upon threshold a frame should have, from receiving the input to displaying the frame on the \acrfull{hmd}, to avoid inducing motion sickness \parencite{valvevrlatency}. One such solution is a low latency control loop that streams \acrshort{vr} scenes containing only the user’s \acrfull{fov} and a latency adaptive margin area around the \acrshort{fov}. This allows the clients to render locally at a high refresh rate to accommodate and compensate for the head movements before the next motion update arrives. \parencite{mvr}. Another angle of attack  leverage's the power of parallel rendering, encoding, transmission and decoding, together with a Remote VSync Driven Rendering approach to minimize \acrshort{mtp} latency \parencite{cutcord}. The prototype for this experiment was based on commodity hardware, which further demonstrates the feasibility of cloud \acrshort{vr} streaming. Lastly a technique known as 'Adaptive \acrfull{fov}' was explored by a multitude of research papers. In essence the optimization is to send only what the user sees (their \acrshort{fov}) and an adaptive area around it, to facilitate for local head movement before the next frame arrives. The idea of rendering only what the user has to see to keep up the immersion is well established within the game development community. View Frustum culling and Occlusion culling \parencite{cullingdefinition} are widely used in games to increase performance, whereas Adaptive \acrshort{fov} aims to decrease latency by reducing the payload of network transmissions.

%\subsection{Older ($>$5 years old) research and technology} 
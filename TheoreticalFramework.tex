\section{Theoretical Framework}

In order to thoroughly understand the aim and subject of the research, it is important to explore different existing solutions and literature. Therefore, the subjects that will be discussed in the  following theoretical framework are Cloud Streaming/Cloud Computing and Virtual reality. Within this theoretical framework, definitions of the subjects will be given as well as current insights into these subjects. The topics reflect knowledge needed to understand the problem space. Together all of the topics make up the 360 scan.

\subsection{Cloud Streaming/Cloud Computing}

\subsubsection{Definition}
According to Armbrust et al. (2010) Cloud computing is defined as follows: 
"Cloud computing refers to both the applications delivered as services over the Internet and the hardware and systems software in the data centers that provide those services." \parencite[]{aviewoncc}
We can then further define Cloud streaming as the applications that are delivered over the internet as a service.

\subsubsection{Existing Solutions and Technology}
Several commercial gaming Cloud streaming services already exist, such as Google Stadia \parencite{stadia}, XBox XCloud \parencite{xcloud} and Nvidia GeForceNow \parencite{geforcenow}. There is also a variety of Infrastructure\hyp{}as\hyp{}a\hyp{}service (IaaS) platforms, such as Amazon's AWS \parencite{aws}, Microsoft's Azure \parencite{azure} and Google's Cloud Platform \parencite{gcp}. 

\subsection{Virtual Reality}

\subsubsection{Definition}
Early works, such as the 1996 article "Virtual Reality in Scientific Visualizations"  or the 2003 book "Virtual Reality Technology", define Virtual Reality as: "The use of computers and human\hyp{}computer interfaces to create the effect of a three\hyp{}dimensional world containing interactive objects with a strong sense of three\hyp{}dimensional presence."\parencite{vrsv} or "A Simulation in which computer graphics is used to create a realistic looking world. Moreover, the synthetic world is not static, but responds to the users input. This defines a key feature of Virtual Reality, \textit{real\hyp{}time interactivity}" \parencite{vrtech}.
\subsubsection{Latency Constraints}